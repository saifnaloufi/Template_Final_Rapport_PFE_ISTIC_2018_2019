\chapter*{Conclusion Général}
Le présent document est une présentation du travail réalisé durant notre stage de fin d'études
au sein de l'entreprise Tritux. Nous avons débuté par comprendre le contexte général du
projet et les différentes exigences du futur système. Nous avons préparé, par la suite, un planning de
travail en respectant les priorités des besoins déjà fixés  \\ \\ Malgré les contraintes de
temps et les difficultés techniques que nous avons rencontré qui se résument principalement dans la
complexité du projet, nous avons réussi à réaliser la totalité de « Gestion des ressources Cloud par projet et budget »\\ \\ Le travail dans le cadre de ce projet de fin d'études, était d'une importance considérable dans la mesure où il nous
a servi comme portail vers le monde professionnel et la vie d'entreprise. \\ \\ De point de vue technique,
il nous a permis de mettre en \oe uvre les acquis théoriques que nous avons appris tout au long de
notre cursus universitaire et de les enrichir. Outre, ce projet était aussi enrichissant pour les bonnes
pratiques de la gestion de projet vu que nous avons eu l'opportunité d'organiser son déroulement
dès le début.\\ \\Loin du gain académique, ce stage nous a permis de mesurer notre capacité à apprendre et à entreprendre dans un court délai.\\ \\
Finalement, notre travail ne s'arrête pas à ce niveau. En effet, plusieurs perspectives s'offrent à notre
application.
Parmi les fonctionnalités que nous pouvons envisager pour « Gestion des ressources Cloud par projet et budget » :

\begin{itemize}
	\item S'ouvrir à plusieurs fournisseurs Cloud  tels que AWS et Azure.
	\item Rajouter un volet de estimation du gain pour chaque planning utilisé.
	\item Rajouter un dashboard pour offrir une vue globale sur les différentes fonctionnalités.
\end{itemize}
