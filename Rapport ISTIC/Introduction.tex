
\chapter*{Introduction Générale}
\markboth{Introduction Générale}{} %pour afficher l'entete
\addcontentsline{toc}{chapter}{Introduction Générale}
Les solutions technologiques autour des infrastructures distribuées n'arrêtent pas de progresser, les sociétés consommatrices deviennent de plus en plus nombreux, ils sont aujourd'hui au centre
de tous ces changements.\\\\ Actuellement, une nouvelle tendance a fait son apparition dans le monde des technologies
de l'information. Il s'agit du «Cloud computing». Ce
modèle a récemment émergé comme un nouveau paradigme pour l'hébergement
et la prestation de services sur Internet. Le Cloud computing est le nouveau pas
dans l'évolution des services web et des produits informatiques à la demande.\\\\
A ce niveau, le Cloud computing se déplace comme un remède offrant une architecture distante dont la gestion est garantie par une tierce partie. Le fournisseur de cette architecture garantit par conséquent l'utilisation et le suivi des services à travers des plateformes tel que Google Cloud Plateform  pour le fournisseur Google.\\\\ Bien que  le genre de ces plateformes semble idéal pour la gestion des ressources Cloud. Cependant, ils ne peuvent pas être personnalisées au besoin de l'entreprise. C'est ainsi qu'une mauvaise utilisation des ressources  augmente les dépenses de l'entreprise. Le contrôle des ressources devient  alors impensable pour les entreprises consommatrices du Cloud.\\\\
C'est dans ce cadre que l'équipe Tritux a eu l'idée de créer une
application permettant de gèrer leurs ressources Cloud par projet et budget grâce aux nouvelles technologies de l'information.\\
Ce projet s'est déroulé selon l'esprit de la méthode agile du Scrum ainsi la répartition du rapport va simuler les différents sprints réalisés. Dans cette optique, nous introduisons le plan de ce rapport :\\\\
Nous introduisons dans le premier chapitre le contexte, les notions théoriques liées au projet et ses objectifs à
savoir les fonctionnalités envisagées, Nous présentons de même l'entreprise d'accueil,
ses domaines et la solution actuelle.\\
Le deuxième chapitre sera dédié à l'analyse et les spécifications fonctionnelles et non
fonctionnelles pour la réalisation de notre application.\\
Ensuite, nous entamerons la classification des chapitres selon la méthode Scrum c'est
à dire en sprint et chaque sprint abordera une ou plusieurs fonctionnalités de notre application de l'analyse jusqu'à la réalisation en passant par la conception.\\
Enfin nous terminerons par une conclusion qui établit le bilan du travail effectué et ouvre des
nouvelles perspectives pour améliorer l'application.

